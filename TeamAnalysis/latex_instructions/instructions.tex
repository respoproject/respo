\documentclass[12pt, a4paper]{article}
\usepackage{amsmath}

\begin{document}
	\section{Data models}
	We currently use five models as described below.
	\begin{enumerate}
		\item 
		\textbf{Maximal absolute lack} finds the competence that has the biggest absolute diference.
		$$\max_{\text{competences}}\{|\text{competence score} - \text{required compentence score}|\}.$$
		\item 
		\textbf{Maximal relative lack} finds the competence that has the biggest relative difference.
		$$\max_{\text{competences}}\left\{\left|\frac{\text{required competence score}}{\text{competence score}}\right|\right\}.$$
		\item 
		\textbf{Most important competence that lack} finds the competence that has the biggest required competence score and isn't satisfied.
		\item 
		\textbf{Improve competence by formula} finds the competence that should be improved using weights.
		$$\max_{\text{competences}}\{|\text{required competence score} - \text{competence score}|\cdot\text{competence weight}\}.$$
		\item 
		\textbf{Importance over number} finds the competence that should be improved that is bigger then the given number. It determines which competence should be improved using one of previou models.
	\end{enumerate}
	The names and algorithms for data models are a subject of change.
	\section{User interface}
	The current python code comes with a user interface. It can be used by running the \texttt{userInterface.py} file. 
\end{document}